\documentclass{article}
\usepackage[T1]{fontenc}
%%\usepackage[utf8]{inputenc}
\usepackage[utf8]{inputenc}
\usepackage{amsmath,amssymb}
\usepackage[russian]{babel}


\begin{document}

    По физике всё-таки лучше иметь тетрадку.
    Мда...

Преподаватель - Манаков николай Александрович (?)

    Адрес: manakov2004@mail.ru

    Учебник:
    Т.И Трафимова - "Курс физики"
    
    Для получения зачёта нужно выполнять лабораторные работы. Их будут
    проводить другие преподаватели (два)

    Необходимо вести/состовлять коспекты
    
    Необходимо отчитываться об выполненых лабораторных работах

    Основый работы компьютера, естественно является движение электронов.
    Существует три способа описания этого движения.

    Интересно посмотреть про оптический компьютер
    
    Разница оптики и электронов:
    первое - бозон
    второе - фермион

    И из-за этой разницы они ведут себя координально иначе
    

    \paragraph
    Классическая механика, основанная на законах Ньютона

    Квантовая механика, основанная на уровнении Шрёдингера.

    Релятивистская механика, основанная на теории относительности.



    Определения:
    1. Материальная точка 
    - это обьект, в рамках решения задач размером которого можно принебречь
    2. Абсолютно твёрдое тело
    - это тело, любые точки внутри которого не меняют расстояния между собой при любых условиях
    3. Поступательное движение
    - это движение, при котором любая прямая жёстко связанная с движущимся телом остаётся параллельной своему первоначальному положени.
    4. Вращательное движение 
    - это движение, при котором все точки тела движутся по окружностям, центры которых дежат на одной и той же прямой, называемой осью вращения.
    5. Система отчёта
    - совокупность системы координат и часов, связанных с оределённой точкой пространства
    
    $ x = x(t), y = y(t), z = z(5) $ 

    $ \overrightarrow{r} = \overrightarrow{r}(t) $

    тут будет рисунок

    Скорость 

    $<r> = \frac{\delta \overrightarrow{r}}{\delta r}$
    
    $$ |\overrightarrow(v)|
    = | \lim\limits{\Delta r \to 0} \delta r \delta t | 
    = \lim\limits{\Delta r \to 0} \frac{|\delta r \overrightarrow{t}|}{\delta t}
    $$

    Прямолинейное равномерное движение 
    
    %$
   %a\_t = 0
   %a\_n = 0

   %a\_t = a = const a_n = 0

   %a\_t = a = \frac{\Delta v}{\Delta t} = \frac{v_2 - v_1}{t_1 - t_2}
    %$

    $ 
    \overrightarrow{\omega} = \lim\limits{\Delta t \to 0}
    $

    $
    T = 2\pi/\omega
    $

    $
    n = 1/T = \omega/(2\pi)

    \omega = 2\pi / n
    $

    Что-то про угловую скорость, вектор ускорения и противоположное
    направление угловой скорости
    Направлено по оси вращения? что?

    Ахуеть ещё больше уравнений 
    и производные...

    В итоге получается разница между нормальным ускорением и угловой скоростью
    
    
### Динамика 
- Динамика описывает причины движения

Ньютон - основатель классической механики

Был директором монетного двора, оказывается. Были расстройства, связанные с алхимией. 

Динамика материальной точки и поступательного движения твёрдого тела

Первый закон ньютона (закон инерции)
Всякая материальная точка (тело) сохраняет состояния покоя или равномерного прямолинейного движения до тех пор, пока воздействия со стороны других тел не заставит её изменить это состояние. 

Первый закон Ньютона удтверждает существование инерциалных систем отсчёта.
Первый закон Ньютона выполняется в инерциальных системах отсмчёта.

Стремление тела сохранять состояние покоя или равномерного прямолинейного движения называется инертностью.

Любая точка, связанная с поверхностью земли, как и сама земля, не является инерциальное системой отсчёта.
Невозможно во вселенной найти объект, системой остчёта которой будет инерциальная система. 
Но приблежонно считать это можно, при искючении трения

Масса тела - физическая величина, определяюзая её инерционные (инертная масса) и гравитационные (гравитационная масса) свойства 

Сила - эт овекторная величика, являюзаяся терой механического воздействия на тело со стороны других тел или полей, в результате которого тело приобретает ускорение или зименяет свою орму и размеры 

Второй закон ньютона: ускорение приобретаемое тематериальной точкой или телом пропорционально вывывающей его силе, совпадает с неею по направлению и обратно пропорционально массе материальной точки (тела)
\end{document}
